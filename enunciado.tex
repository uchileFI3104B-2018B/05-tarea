\documentclass[letter, 11pt]{article}
%% ================================
%% Packages =======================
\usepackage[utf8]{inputenc}      %%
\usepackage[T1]{fontenc}         %%
\usepackage{lmodern}             %%
\usepackage[spanish]{babel}      %%
\decimalpoint                    %%
\usepackage{fullpage}            %%
\usepackage{fancyhdr}            %%
\usepackage{graphicx}            %%
\usepackage{amsmath}             %%
\usepackage{color}               %%
\usepackage{mdframed}            %%
\usepackage[colorlinks]{hyperref}%%
%% ================================
%% ================================

%% ================================
%% Page size/borders config =======
\setlength{\oddsidemargin}{0in}  %%
\setlength{\evensidemargin}{0in} %%
\setlength{\marginparwidth}{0in} %%
\setlength{\marginparsep}{0in}   %%
\setlength{\voffset}{-0.5in}     %%
\setlength{\hoffset}{0in}        %%
\setlength{\topmargin}{0in}      %%
\setlength{\headheight}{54pt}    %%
\setlength{\headsep}{1em}        %%
\setlength{\textheight}{8.5in}   %%
\setlength{\footskip}{0.5in}     %%
%% ================================
%% ================================

%% =============================================================
%% Headers setup, environments, colors, etc.
%%
%% Header ------------------------------------------------------
\fancypagestyle{firstpage}
{
  \fancyhf{}
  \lhead{\includegraphics[height=4.5em]{LogoDFI.jpg}}
  \rhead{FI3104-1 \semestre\\
         Métodos Numéricos para la Ciencia e Ingeniería\\
         Prof.: \profesor}
  \fancyfoot[C]{\thepage}
}

\pagestyle{plain}
\fancyhf{}
\fancyfoot[C]{\thepage}
%% -------------------------------------------------------------
%% Environments -------------------------------------------------
\newmdenv[
  linecolor=gray,
  fontcolor=gray,
  linewidth=0.2em,
  topline=false,
  bottomline=false,
  rightline=false,
  skipabove=\topsep
  skipbelow=\topsep,
]{ayuda}
%% -------------------------------------------------------------
%% Colors ------------------------------------------------------
\definecolor{gray}{rgb}{0.5, 0.5, 0.5}
%% -------------------------------------------------------------
%% Aliases ------------------------------------------------------
\newcommand{\scipy}{\texttt{scipy}}
%% -------------------------------------------------------------
%% =============================================================
%% =============================================================================
%% CONFIGURACION DEL DOCUMENTO =================================================
%% Llenar con la información pertinente al curso y la tarea
%%
\newcommand{\tareanro}{5}
\newcommand{\fechaentrega}{1/11/2018 23:59 hrs}
\newcommand{\semestre}{2018B}
\newcommand{\profesor}{Valentino González}
%% =============================================================================
%% =============================================================================


\begin{document}
\thispagestyle{firstpage}

\begin{center}
  {\uppercase{\LARGE \bf Tarea \tareanro}}\\
  Fecha de entrega: \fechaentrega
\end{center}


%% =============================================================================
%% ENUNCIADO ===================================================================
\noindent{\large \bf Problema 1}

El oscilador de van der Pool fue propuesto para describir la dinámica de
algunos circuitos eléctricos. La ecuación es la siguiente:

$$\frac{d^2x}{dt^2} = - k x - \mu (x^2 - a^2) \frac{dx}{dt}$$

donde $k$ es la constante elástica y $\mu$ es un coeficiente de roce. Si $|x| >
a$ el roce amortigua el movimiento, pero si $|x| < a$ el roce inyecta energía.
Se puede hacer un cambio de variable para convertir la ecuación a:

$$\frac{d^2y}{ds^2} = - y - \mu^* (y^2 - 1) \frac{dy}{ds}$$

con lo cual ahora la ecuación sólo depende de un parámetro, $\mu^*$. Indique
cuál es el cambio de variable realizado.

Integre la ecuación de movimiento usando el método de Runge-Kutta de orden 4
visto en clase. Se pide que Ud. implemente su propia versión del algoritmo
(escriba un algoritmo genérico para que pueda re-usarlo en el Problema 2 de
esta misma tarea). Describa la discretización usada y el paso de tiempo. Use
$\mu^*=1.RRR$, donde $RRR$ son los 3 últimos dígitos de su RUT antes del guión.

Integre la solución hasta $T=20\pi$ (aproximadamente 10 períodos) para las
siguientes condiciones iniciales:

\begin{align*}
1) \frac{dy}{ds} = 0; y = 0.1\\
2) \frac{dy}{ds} = 0; y = 4.0
\end{align*}

Grafique $y(s)$ y la trayectoria en el espacio $(y, dy/ds)$

\vspace{1.5em}
\noindent{\large \bf Problema 2}

Considere un péndulo simple que es forzado periódicamente, de manera que su
movimiento es descrito porla siguiente ecuación:

$$ m L^2 \ddot{\phi} = -mgLsin(\phi) + F_0 cos(\omega t) $$

La frecuencia natural de pequeñas oscilaciones del péndulo es
$\omega_0=\sqrt{g/L}$ pero para oscilaciónes más grandes, la frecuencia es
menor. Se espera, por lo tanto, que el péndulo entre en resonancia para
frecuencias de forzamiento levemente menores que $\omega_0$.

Se pide que Ud. integre numéricamente la ecuación de movimiento y determine
cuál es la frecuencia de forzamiento para la cual el péndulo alcanza la máxima
amplitiud (después de oscilar muchas veces).

\vspace{0.5em}
\noindent\underline{Indicaciones}

\begin{itemize}
  \item Considere la condición inicial $\phi(0) = \dot{\phi}(0) = 0$.
  \item Use el método de RK4 implementado para el Problema 1 con un paso
    temporal $\Delta t$ menor a un centésimo del período natural del péndulo.
    Explore qué valores funcionan mejor para el paso temporal.
  \item Considere los siguientes valores para los parámetros del problema:
    \begin{align*}
      m = 0.85 * 1.0RRR \\
      L = 1.75 * 1.0RRR \\
      F_0 = 0.05 * 1.0RRR
    \end{align*}
\end{itemize}

donde $RRR$ son los 3 últimos dígitos de su RUT.

Explique en su informe cuál fue su estrategia para encontrar el valor de la
frecuencia de forzamiento que produce la máxima amplitud.

\begin{ayuda}
  \small
  No olvide incluir su Nombre y RUT en el informe.
\end{ayuda}


\vspace{1em}
\noindent{\bf Instrucciones importantes.}
\begin{itemize}

  \item Utilice \texttt{git} durante el desarrollo de la tarea para mantener un
    historial de los cambios realizados. La siguiente
    \href{https://education.github.com/git-cheat-sheet-education.pdf}{cheat
      sheet} le puede ser útil. \textbf{Revisaremos el uso apropiado
    de la herramienta y asignaremos una fracción del puntaje a este ítem.}
    Realice cambios pequeños y guarde su progreso (a través de \emph{commits})
    regularmente. No guarde código que no corre o compila (si lo hace por algún
    motivo deje un mensaje claro que lo indique). Escriba mensajes claros que
    permitan hacerse una idea de lo que se agregó y/o cambió de un
    \texttt{commit} al siguiente.

  \item Revisaremos su uso correcto de \texttt{python}. Si define una función
    relativametne larga o con muchos parámetros, recuerde escribir el
    \emph{docstring} que describa los parámetros que recibe la función, el
    output, y el detalle de qué es lo que hace la función. Recuerde que
    generalmente es mejor usar varias funciones cortas (que hagan una sola cosa
    bien) que una muy larga (que lo haga todo).  Utilice nombres explicativos
    tanto para las funciones como para las variables de su código.  El mejor
    nombre es aquel que permite entender qué hace la función sin tener que leer
    su implementación.

  \item También evaluaremos que su código apruebe la guía de estilo sintáctico
    \href{https://www.python.org/dev/peps/pep-0008/}{\texttt{PEP8}}. En
    \href{http://pep8online.com}{esta página} puede chequear si su código
    aprueba \texttt{PEP8}.

  \item La tarea se entrega subiendo su trabajo a github. Clone este
    repositorio (el que está en su propia cuenta privada), trabaje en el código
    y en el informe y cuando haya terminado asegúrese de hacer un último
    \texttt{commit} y luego un \texttt{push} para subir todo su trabajo a
    github.

  \item El informe debe ser entregado en formato \texttt{pdf}, este debe ser
    claro sin información de más ni de menos. \textbf{Esto es muy importante,
    no escriba de más, esto no mejorará su nota sino que al contrario}. La
    presente tarea probablemente no requiere informes de más de 5 páginas en
    total (sólo una referencia útil).  Asegúrese de utilizar figuras efectivas
    y tablas para resumir sus resultados. Revise su ortografía.

  \item No olvide indicar su RUT en el informe.

  \item Repartición de puntaje: 40\% implementación y resolución de los
    problemas (independiente de la calidad de su código -- 20\% cada problema);
    45\% calidad del reporte entregado: demuestra comprensión del problema y su
    solución, claridad del lenguaje, calidad de las figuras utilizadas; 5\%
    aprueba a no \texttt{PEP8}; 10\% diseño del código: modularidad, uso
    efectivo de nombres de variables y funciones, docstrings, \underline{uso
    efectivo de git}, etc.

\end{itemize}


%% FIN ENUNCIADO ===============================================================
%% =============================================================================

\end{document}
