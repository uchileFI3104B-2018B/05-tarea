\documentclass[letter, 11pt]{article}
%% ================================
%% Packages =======================
\usepackage[utf8]{inputenc}      %%
\usepackage[T1]{fontenc}         %%
\usepackage{lmodern}             %%
\usepackage[spanish]{babel}      %%
\decimalpoint                    %%
\usepackage{fullpage}            %%
\usepackage{fancyhdr}            %%
\usepackage{graphicx}            %%
\usepackage{amsmath}             %%
\usepackage{color}               %%
\usepackage{mdframed}            %%
\usepackage[colorlinks]{hyperref}%%
%% ================================
%% ================================

%% ================================
%% Page size/borders config =======
\setlength{\oddsidemargin}{0in}  %%
\setlength{\evensidemargin}{0in} %%
\setlength{\marginparwidth}{0in} %%
\setlength{\marginparsep}{0in}   %%
\setlength{\voffset}{-0.5in}     %%
\setlength{\hoffset}{0in}        %%
\setlength{\topmargin}{0in}      %%
\setlength{\headheight}{54pt}    %%
\setlength{\headsep}{1em}        %%
\setlength{\textheight}{8.5in}   %%
\setlength{\footskip}{0.5in}     %%
%% ================================
%% ================================

%% =============================================================
%% Headers setup, environments, colors, etc.
%%
%% Header ------------------------------------------------------
\fancypagestyle{firstpage}
{
  \fancyhf{}
  \lhead{\includegraphics[height=4.5em]{LogoDFI.jpg}}
  \rhead{FI3104-1 \semestre\\
         Métodos Numéricos para la Ciencia e Ingeniería\\
         Prof.: \profesor}
  \fancyfoot[C]{\thepage}
}

\pagestyle{plain}
\fancyhf{}
\fancyfoot[C]{\thepage}
%% -------------------------------------------------------------
%% Environments -------------------------------------------------
\newmdenv[
  linecolor=gray,
  fontcolor=gray,
  linewidth=0.2em,
  topline=false,
  bottomline=false,
  rightline=false,
  skipabove=\topsep
  skipbelow=\topsep,
]{ayuda}
%% -------------------------------------------------------------
%% Colors ------------------------------------------------------
\definecolor{gray}{rgb}{0.5, 0.5, 0.5}
%% -------------------------------------------------------------
%% Aliases ------------------------------------------------------
\newcommand{\scipy}{\texttt{scipy}}
%% -------------------------------------------------------------
%% =============================================================
%% =============================================================================
%% CONFIGURACION DEL DOCUMENTO =================================================
%% Llenar con la información pertinente al curso y la tarea
%%
\newcommand{\tareanro}{5}
\newcommand{\fechaentrega}{1/11/2018 23:59 hrs}
\newcommand{\semestre}{2018B}
\newcommand{\profesor}{Valentino González}
%% =============================================================================
%% =============================================================================


\begin{document}
\thispagestyle{firstpage}

\begin{center}
  {\uppercase{\LARGE \bf Tarea \tareanro}}\\
  Fecha de entrega: \fechaentrega
\end{center}


%% =============================================================================
%% ENUNCIADO ===================================================================
\noindent{\large \bf Problema 1}

El oscilador de van der Pool fue propuesto para describir la dinámica de
algunos circuitos eléctricos. La ecuación es la siguiente:

$$\frac{d^2x}{dt^2} = - k x - \mu (x^2 - a^2) \frac{dx}{dt}$$

donde $k$ es la constante elástica y $\mu$ es un coeficiente de roce. Si $|x| >
a$ el roce amortigua el movimiento, pero si $|x| < a$ el roce inyecta energía.
Se puede hacer un cambio de variable para convertir la ecuación a:

$$\frac{d^2y}{ds^2} = - y - \mu^* (y^2 - 1) \frac{dy}{ds}$$

con lo cual ahora la ecuación sólo depende de un parámetro, $\mu^*$. Indique
cuál es el cambio de variable realizado.

Integre la ecuación de movimiento usando el método de Runge-Kutta de orden 4
visto en clase. Se pide que Ud. implemente su propia versión del algoritmo
(escriba un algoritmo genérico para que pueda re-usarlo en el Problema 2 de
esta misma tarea). Describa la discretización usada y el paso de tiempo. Use
$\mu^*=1.RRR$, donde $RRR$ son los 3 últimos dígitos de su RUT antes del guión.

Integre la solución hasta $T=20\pi$ (aproximadamente 10 períodos) para las
siguientes condiciones iniciales:

\begin{align*}
1) \frac{dy}{ds} = 0; y = 0.1\\
2) \frac{dy}{ds} = 0; y = 4.0
\end{align*}

Grafique $y(s)$ y la trayectoria en el espacio $(y, dy/ds)$

\vspace{1em}
\noindent{\large \bf Problema 2}

Considere un péndulo simple que es forzado periódicamente, de manera que su
movimiento es descrito porla siguiente ecuación:

$$ m L^2 \ddot{\phi} = -mgLsin(\phi) + F_0 cos(\omega t) $$

La frecuencia natural de pequeñas oscilaciones del péndulo es
$\omega_0=\sqrt{g/L}$ pero para oscilaciónes más grandes, la frecuencia es
menor. Se espera, por lo tanto, que el péndulo entre en resonancia para
frecuencias de forzamiento levemente menores que $\omega_0$.

Se pide que Ud. integre numéricamente la ecuación de movimiento y determine
cuál es la frecuencia de forzamiento para la cual el péndulo alcanza la máxima
amplitiud (después de oscilar muchas veces).

\vspace{0.5em}
\noindent\underline{Indicaciones}

\begin{itemize}
  \item Considere la condición inicial $\phi(0) = \dot{\phi}(0) = 0$.
  \item Use el método de RK4 implementado para el Problema 1 con un paso
    temporal $\Delta t$ menor a un centésimo del período natural del péndulo.
    Explore qué valores funcionan mejor para el paso temporal.
  \item Considere los siguientes valores para los parámetros del problema:
    \begin{align*}
      m = 0.85 * 1.0RRR \\
      L = 1.75 * 1.0RRR \\
      F_0 = 0.05 * 1.0RRR
    \end{align*}
\end{itemize}

donde $RRR$ son los 3 últimos dígitos de su RUT.

Explique en su informe cuál fue su estrategia para encontrar el valor de la
frecuencia de forzamiento que produce la máxima amplitud.

\begin{ayuda}
  \small
  No olvide incluir su Nombre y RUT en el informe.
\end{ayuda}





%% FIN ENUNCIADO ===============================================================
%% =============================================================================

\end{document}
